\documentclass[a4paper,12pt]{article}
\usepackage[utf8]{inputenc}
\usepackage[spanish]{babel}
\usepackage{graphicx}
\begin{document}
\includegraphics[width=0.5\textwidth]{imagen1.eps}
\title{Título del artículo}
  \author{Nombre y Apellido \\
          Técnicas Experimentales~\footnote{Universidad de La Laguna}
         }
  \date{\today}
  \maketitle
   \begin{abstract}
     En \LaTeX{}~\cite{Lam:86} es sencillo escribir expresiones
     matemáticas como $a=\sum_{i=1}^{10} {x_i}^{3}$
     y deben ser escritas entre dos símbolos \$.
     Los superíndices se obtienen con el símbolo \^{}, y
     los subíndices con el símbolo \_.
     Por ejemplo: $x^2 \times y^{\alpha + \beta}$.
     También se pueden escribir f\'ormulas centradas:
   \[h^2=a^2 + b^2 \]
 \end{abstract}
 \section{Primera sección}
Si simplemente se desea escribir un texto normal en latex,
sin complicadas f\'ormulas matem\'aticas o efectos especiales
como cambios  de fuente, entonces simplemente tiene que escribir
en espa\~nol normalmente.
\par
Si desea cambiar el p\'arrafo ha de dejar una l\'inea en blanco o bien
utilizar el comando \verb|\par|.

No es necesario preocuparse de la sangr\'ia de los p\'arrafos:
todos los p\'arrafos se sangrar\'an autom\'aticamente con la excepi\'on
del primer p\'arrafo de una secci\'on.

Se ha de distinguir entre las comillas simples 'izquierdad' 
y la comilla simple 'derecha' cuando se escribe en el ordenador.

En el caso que se quiera utilizar comillas dobles se han de
escribir dos caracteres 'comillas simples' seguidos, estos es,
"comillas dobles".

\bigskip
  \begin{tabular}{|l|c|c|}
   \hline
   Nombre & Edad & Nota \\ \hline
   Pepe & 24 & 10 \\ \hline
   Juan & 19 & 8 \\ \hline
   Luis & 21 & 9 \\ \hline
   \end{tabular}
Tambi\'en se ha de tener cuidado con los guiones: se utiliza un \'unico
guin para la separaci\'on de s\'ilabas, mientras que se utilizan 
tres  guiones seguidos para producir un gui\'on de los que se usan
como signo de puntuaci\'on --- como en esta oraci\'on.
\begin{thebibliography}{00}
   \bibitem{Lam:86}
      Lamport, Leslie.
      TLA in pictures.
      \emph{IEEE Transactions on Software Engineering},
      21(9), 768-775.
      (1995)
\end{thebibliography}
\end{document}
\end{document}

